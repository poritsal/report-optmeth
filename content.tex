\section*{Введение}
На сегодняшний день задача коммивояжера - одна из самых известных в области теории вычислений и комбинаторной оптимизации. Будучи сформулированной ещё в XX веке, она до сих пор не имеет оптимального решения, что непременно делает её актуальной темой для исследования.\\ 

Задача коммивояжера (TSP, travelling salesman problem) заключается в поиске оптимального пути между городами и находит применение во многих отраслях науки и промышленности. Так, примером может служить оптимизация севооборота и уборочного процесса  [\ref{bib7}], задачи планирования работы транспорта [\ref{bib8}], организации работы станков с программным управлением [\ref{bib9}] и так далее.\\

Рассматриваемая задача относится к классу NP-полных. Сегодня не существует алгоритма, который позволил бы решить её за полиномиальное время. В общем случае, для вычисления результата при $n$ элементах потребуется $n!$ операций, то есть время выполнения составит $O(n!)$ [\ref{bib10}]. Это значит, что при количестве городов, хоть сколько-нибудь приближенному к их числу в практических условиях, задачу не получится решить методом перебора за время меньшее, чем оставшееся время жизни Солнца.\\

Таким образом, актуальная задача, не имеющая эффективного по времени решения, остаётся камнем преткновения для учёных по всему миру. Предложены различные методы нахождения оптимального пути странствующего торговца, среди которых, например, методы ветвей и границ, имитации отжига, эластичной сети и другие. \\

В данной работе рассматриваются два пути решения: метод динамического программирования и муравьиный алгоритм, которые будут рассмотрены позднее в разделе 1.2. Целью данной работы является сравнение временных и пространственных затрат при решении поставленной задачи разными способами.

\addcontentsline{toc}{section}{Введение}
\section{Обзор задачи коммивояжёра с усложнением}
\subsection{Общая характеристика задачи коммивояжёра}
\subsubsection{Постановка задачи коммивояжёра}
Даны $n$ городов одной группы, каждый из которых имеет заданный рейтинг $r$. Путь между соседними городами занимает время $t$ (для каждого пути своё). Найти такой путь, который за заданное время обойдёт города с суммарным рейтингом не меньше заданного. В случае, если условиям удовлетворяют несколько маршрутов, представить их все. В ходе работы необходимо исследовать и сравнить методы решения поставленной задачи.


\subsubsection{Формализация задачи коммивояжёра}
Введём обозначения:\\
Пусть $n\geq{0}$ - количество городов.\\
$G(V,E)$ - граф, где $V$ - множество вершин(городов), $E$ - множество рёбер(дорог между городами).\\
$C = \{c_{ij}\}_{n \times n}\;\; (c(u, v)\rightarrow\mathbb{R})$ - матрица времени перемещения между городами; $u, v \in V$.\\
$t_{lim}$ - заданное ограничение по времени.\\
$r_{min}$ - заданный рейтинг, который нужно набрать по прохождении нескольких городов.\\
$r(u)\rightarrow \mathbb{R}$ - рейтинг города u.\\
Пусть $W$ - множество всех циклов.\\
Тогда множество допустимых точек имеет вид:\\
$$\Omega = \{\omega \in W | \sum_{(u,v)\in \omega}c(u, v)\leq{t_{lim}},\;\; \sum_{u \in W} r(u)\geq{r_{min}}\}$$

Стоит отметить, что данная задача не имеет целевой функции, а ставит своей целью нахождение именно множества допустимых точек, так как в качестве ответа необходимо представить все маршруты, удовлетворяющие поставленным условиям.

%Введём обозначения:\\
%Пусть $n\geq{0}$ - количество городов\\
%$T\geq{0}$ - ограничение по времени\\
%$R\geq{0}$ - заданный рейтинг\\
%$M_{n}$ - вектор рейтинга городов\\
%$G(V,E)$ - граф, где $V$ - множество вершин(городов), $E$ - множество рёбер(дорог между городами).\\
%$C = \{c_{ij}\}_{n \times n}$ - матрица времени перемещения между городами.\\

%Тогда множество допустимых точек:\\
%$$m_{i}\geq{0},\  i = \overline{1,n}$$\\
%$$c_{ij}\geq{0},\  i = \overline{1,n},\  j = \overline{1,n}$$\\
%$$\sum\limits_{i = 0}^{k}m_{i}\geq{R},\ \ k\in[0, n]$$\\
%$$\sum\limits_{i,j = 0}^{n}c_{ij}\leq{T}$$\\

%Целевая функция:\\
%$$\sum\limits_{i = 0}^{k}m_{i}\rightarrow{max},\ k\in[0, n]$$


%\begin{equation*}
    %\left\{
        %\begin{array}{ccl}
            %\sum\limits_{i = 0}^{k}m_{i}\rightarrow{max},\ k\in[0, n]\\
            %\sum\limits_{i = 0}^{k}m_{i}\geq{R},\ k\in[0, n]
        %\end{array}
        %\right.
%\end{equation*}




\subsection{Алгоритмы решения задачи коммивояжёра}
Методы решения задачи коммивояжёра можно условно разделить на две группы: точные и приближённые. Хотя первые позволяют получить решение с наибольшей точностью, их использование на практике осложняется алгоритмической сложностью и чувствительностью к специфике задачи и дополнительным условиям. Это влечёт за собой появление эвристических методов, которые чаще всего используются на практике [\ref{bib9},\ref{bib11}]. Эвристики представляют собой попытку учесть специфику задачи простыми средствами [\ref{bib11}].\\

В этой работе исследуются и сравниваются методы-представители обеих групп. Метод динамического программирования - точный, а муравьиный алгоритм - эвристический.

\subsubsection{Динамическое программирование}
\subsubsection{Муравьиный алгоритм}
\section{Решение задачи коммивояжёра}
\subsection{Решение методом динамического программирования}
\subsection{Решение муравьиным алгоритмом}
\subsection{Сравнение метода динамического программирования и муравьиного алгоритма при решении задачи коммивояжёра с усложнением}
\subsubsection{Сравнение временных затрат}
\subsubsection{Сравнение пространсвенной сложности}
\section*{Заключение}
\addcontentsline{toc}{section}{Заключение}

\newpage
\section*{Список используемой литературы}
\addcontentsline{toc}{section}{Список используемой литературы}

\begin{enumerate}
    \item \label{bib1} Задача коммивояжера (TSP) точное решение — метод динамического программирования -- Хабр // \url{https://habr.com}. URL: \url{https://habr.com/ru/articles/701458/}
    
    \item \label{bib2} Мудров В.И. Задача о коммивояжёре. — М.: «Знание», 1969. — С. 62. 
    
    \item \label{bib3} Муравьиные алгоритмы -- Хабр // \url{https://habr.com}. URL: \url{https://habr.com/ru/articles/105302/}
    
     \item \label{bib4} Томас Х. Кормен, Чарльз И. Лейзерсон, Рональд Л. Ривест, Клиффорд Штайн. Алгоритмы: построение и анализ = Introduction to Algorithms. — 2-е изд. — М.: «Вильямс», 2006. — С. 1296. — ISBN 0-07-013151-1.
     
    \item \label{bib5} Штовба С.Д. Муравьиные алгоритмы // Exponenta Pro. Математика в приложениях. – 2003. – No4. – С. 70–75

    \item \label{bib6} M. Dorigo \& T. Stützle, 2004. Ant Colony Optimization, MIT Press. ISBN 0-262-04219-3

    \item \label{bib7} Ананич И.Г., Захарова В. С., Толкач Г. В. Задача коммивояжера и её применение в сельском хозяйстве, сельскохозяйственный журнал, 2016.

    \item \label{bib8} Лябах. Н. Н., Казак А. А. Применение и развитие задачи коммивояжера для решения технологических задач на железнодорожном транспорте - Вестник Ростовского государственного университета путей сообщения - №1 - 2002 - с. 141 - 143.

    \item \label{bib9} Бородин В. В., Ловецкий С. Е., Меламед И. И., Плотинский Ю. М. Экспериментальное
    исследование эффективности эвристических алгоритмов решения задачи коммивояжёра //Автоматика и телемеханика. – 1980. – No. 11.

    \item \label{bib10} А. Бхаргава. Грокаем алгоритмы. Иллюстрированное пособие для программистов и любопытствующих. - Спб.: Питер, 2022. - 288с.:ил. - (Серия <<Библиотека программиста>>). ISBN 978-5-4461-0923-4. 

    \item \label{bib11} Меламед И. И., Сергеев С. И., Сигал И. Х., Задача коммивояжёра. Приближенные
    алгоритмы//Автоматика и телемеханика. – 1989. – No. 11.
\end{enumerate}