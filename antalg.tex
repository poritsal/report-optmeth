\subsection{Описание метода}
Муравьиный алгоритм основывается на моделировании поведения муравьёв, которые осуществляют поиск пути от муравейника до источника пищи, за что он и получил своё название [\ref{bib5}]. Если муравей находит источник пищи, то возвращается в гнездо, оставляя за собой испаряющийся след из феромонов. Эти феромоны привлекаяют других муравьёв, которые вероятнее всего пойдут по этому маршруту, а на обратном пути укрепят феромонную тропу. Если существует 2 маршрута, то по более короткому, за то же время, успеют пройти больше муравьёв, чем по длинному, а значит, на нём останется больше феромонов. Таким образом, длинные пути, в конечном итоге, исчезнут из-за испарения феромонов; чем больше времени требуется для прохождения пути до цели и обратно, тем сильнее испарится феромонная тропа. На коротком пути, напротив, плотность феромонов остаётся высокой [\ref{bib12}].\\

В общем случае поиск оптимального решения представляет собой итерационный процесс, в котором муравьи одного поколения по очереди проходят возможные маршруты, а выбор муравьём следующего города обуславливается вероятностью попадания в него. Вероятность, в свою очередь, определяется расстоянием до города и количеством феромонов на пути к нему [\ref{bib12}].\\

В нашем случае необходимо учитывать рейтинг города и ограничение по времени, поэтому классический алгоритм можно адаптировать под поставленную задачу следующим образом:\\

\subsection{Модификация метода для поставленной задачи}
Для решения поставленной перед нами задачи, учтем наличие рейтинга у городов, положив его в условие перехода муравья между городами, а также будем стараться его максимизировать. Помимо этого, необходимо учитывать ограничения условий данной задачи, таким образом муравей не может перейти в город, время для переход в который превышает оставшееся время. Как итог имеем следующий алгоритм:
\begin{enumerate}
    \item \textbf{Инициализация параметров:} 
    \begin{itemize}
        \item количество муравьев.
        \item количество поколений, то есть итераций алгоритма.
        \item коэффициенты \(\alpha\), \(\beta\), \(\gamma\), каждый из которых отвечает за увеличение или уменьшение влияния времени перехода, ферамона и рейтинга города соответственно. 
        \item скорость испарения феромона.
        \item начальное значение феромона одинаковое и используемое для инициализации изначальной привлекательности каждого перехода. 
        \item константа, отвечающая за количество феромонов оставляемых каждым муравьём.
    \end{itemize}
А также входных данных, таких как рейтинги городов, времена пути между городами и ограничение по времени.
\item \textbf{Запуск алгоритма:} В цикле по количеству итераций запускаем процесс поиска оптимальных маршрутов. Для каждой итерации создаем заданное количество муравьев, каждый из которых начинает свой маршрут из случайного города.
\item \textbf{Проход муравья по маршруту:} Для каждого муравья случайно выбираем переход из города i в следующий город j, основываясь на вероятностях, рассчитанных на основе феромонов и эвристических данных (рейтинги и времена пути) по следующей формуле:
\[p_{i, j}=\frac{\tau_{i, j}^\alpha \cdot \eta_{i, j}^\beta \cdot r_{i, j}^\gamma}{\sum\tau_{i, j}^\alpha \cdot \eta_{i, j}^\beta \cdot r_{i, j}^\gamma},\]
где \(\tau_{i, j}\) - количество феромонов на ребре i, j;\\
\(\eta_{i, j}=\frac{1}{t_{i,j}}\) - привлекательность ребра i, j, где \(t_{i,j}\) время соответствующего перехода;\\
\(r\) - рейтинг города j.

\item \textbf{Обновление лучшего пути:} После прохода муравья проверяем полученный маршрут на соответствие условию задачи (общее время маршрута не превышает максимального) и обновляем лучший путь, если суммарный рейтинг полученного маршрута превышает существующий максимум.
\item \textbf{Испарение феромонов:} После прохода всех муравьев обновляем феромоны на путях по следующей формуле:
\begin{equation*}
\Delta \tau_{i,j} = 
 \begin{cases}
   \(\frac{Q}{L_{k}}\), &\text{если муравей прошёл по этому ребру}\\
   0, &\text{если муравей не прошёл по этому ребру }
 \end{cases}
\end{equation*}
\[\tau_{i j} \leftarrow (1-p) \cdot \tau_{i j}+\sum_{k=1}^m \Delta \tau_{i j, k}\]
Здесь \(L_k\) – время муравья k, затраченное на путь между городами i и j,  Q - константа, отвечающая за количество феромонов оставляемых муравьем, \(p\) – интенсивность испарения феромона.
\end{enumerate}
Алгоритмическая сложность метода будет зависеть от количества поколений t, количества муравьев в колонии m и непосредственно количества городов n, следовательно, сложность такого метода будет равна \(O(t \cdot m \cdot n^2)\).